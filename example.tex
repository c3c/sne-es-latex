\documentclass[11pt,a4paper]{article}
\usepackage[a4paper]{geometry}
\usepackage{amsmath}
\usepackage{amsfonts}
\usepackage{hyperref}
\setlength{\parindent}{0 cm}
\usepackage{hyperref}
\usepackage{amssymb}
\usepackage{fancyhdr}
\usepackage[dutch]{babel}
\usepackage{graphicx}
\usepackage[margin=10pt,font=small,labelfont=bf,labelsep=endash]{caption}
\usepackage{subfig}
\usepackage{float}
\pagestyle{fancy}
\fancyhead[CO,CE]{Arya Ghodsi}
\author{Arya Ghodsi}
\begin{document}

%\documentclass[11pt,a4paper]{article}
%\usepackage[latin1]{inputenc}
%\usepackage{amsmath}
%\usepackage{amsfonts}
%\setlength{\parindent}{0 cm}
%\usepackage{amssymb}
%\usepackage[dutch]{babel}
%\usepackage{graphicx}
%\usepackage[margin=10pt,font=small,labelfont=bf,labelsep=endash]{caption}
%\addtolength{\topmargin}{-34pt}
%\addtolength{\textheight}{68pt}
%\date{4 april 2011}
%\begin{document}
\begin{titlepage}
\enlargethispage{10cm}
\begin{flushleft}
\begin{figure}[h]
\includegraphics[width=1.0\textwidth]{logobalk-UGent.jpg}
\end{figure}
\end{flushleft}
\begin{center}
\textsc{Faculteit Wetenschappen}
\\[2cm]
\small{Academiejaar 2011-2012}
\\[3cm]
\LARGE{\textsc{Examenproject}}\\[1cm]
\large{Statistische gegevensanalyse}
\\[11.5cm]
\small{Arya \textsc{Ghodsi}}\\
\end{center}
\end{titlepage}

%\end{document}
\newpage

\section*{Artikel 1 - Privacy is Dead}
Wat is privacy? Waarom proberen we het angstvallig te bewaren? \\ 
De auteur probeert hier met karikaturen een aantal pijnpunten van het concept privacy aan te kaarten en tegelijkertijd geeft hij tussen de lijnen impliciet en expliciet aan waar hij naar toe wilt, nl. dat het klassieke definitie van privacy met de vooruitgang van techniek zal vervagen om uiteindelijk in het niets op te gaan. \\ Hij schets op een interessante manier hoe we onze klassieke visie moeten verruimen over hoe we over privacy nadenken, het belangrijkste dat hier naar voren komt is  dat privacy niet per se iets is dat we willen beschermen, het is eerder het assymmetrie in privacy dat ons stoort. Hij stelt dat indien er enige vorm van symmetrie zou aanwezig zijn, dat de mens het helemaal niet zo erg zou vinden om zijn privacy al dan niet geheel op te geven.\\ Men zou kunnen stellen dat de auteur privacy probeert te evalueren vanuit een psychologisch oogpunt van de mens.\\ Verder is het ook volgens de auteur al reeds onmogelijk om onze privacy te beschermen. Hij haalt twee voorbeelden aan, opvragen van informatie via een smartphone en camera-surveillance. Vervolgens probeert hij uit een te zetten hoe men deze data probeert te annonimeseren ten behoeve van privacy, maar reeds duiken de zwakheden hier op, er zit uiteindelijk niets anders op dan te besluiten dat privacy gewoon moeilijk te bewaren is en als oplossing probeert hij uit te leggen waarom we niet per se privacy moeten nastreven, maar eerder in de omgekeerde richting moeten kijken.\\
\\
\textit{Ik vond dit een heel interessant artikel, vooral omdat de visie totaal anders was, het was niet een opsomming van wat privacy is en hoe deze bewaard kan worden. Het stelde privacy voor op een manier waarop ik het nooit had bekeken. Met de opkomst van sociale netwerken is privacy meer dan ooit een vraagstuk waar velen hun tanden op stuk bijten. Het is een mooie visie, maar het heeft ook zijn zwakke punten vind ik. Met de visie dat de auteur heeft evolueren we, vind ik, naar een maatschappij dat op chantage steunt. Immers men gaat uit van het principe "Ik vertel niks over jouw illegale praktijken, alleen maar omdat je weet van mijn misdragingen". \\
Is dit zoveel slechter dan een maatschappij, die steunt op geheimen, is de vraag dan. Ik vind van wel, het is niet per se een kwestie van symmetrie in privacy, iedereen heeft geheimen, het zijn dingen die u karakteriseren. Plots ben je van een unieke persoon gereduceerd tot een entiteit zoals iedereen, geheel vrij te volgen via alle mogelijke manieren.}
\newpage
\section*{Artikel 2 - Web-Scale Multimedia Analysis: Does Content Matter?}
Neem drie datavormen, die in onze dagdagelijkse leven voorkomen: Muziek, video en afbeeldingen. Onderwerp nu deze drie datavormen aan een onderzoek om meer informatie over de content uit te vissen, waaruit verwacht je dan het meeste informatie te kunnen extraheren? Intu\"{i}tief, zoals ook de auteur, zou men verwachten dat men het meeste informatie uit de content van deze datavormen zou kunnen halen. Niks minder blijkt minder waar te zijn! Het blijkt dat de meta-data, de data over de data. \\
Bij de muziek bestanden, ging hij na hoe je best kon besluiten of twee muziek-fragmenten gelijkaardig waren. Hier werden de resultaten van een wetenschappelijke proef waarbij gekeken werd naar de audio-vorm tegenover de keuring van muziekliefhebbers gezet, de muziekliefhebbers konden beter/accurater beoordelen of een muziekfragment gelijkaardig was. \\ De video's was een experiment opgezet door de streaming-gigant Netflix, die een beter rating-systeem wou voor aanbevelingen te maken, het loofde hier 1 miljoen USD uit. Alweer waren er hier opmerkelijke resultaten, het was niet de content van video's maar meta-data zoals de ouderdom van de film, die effectief een betere resultaat opleverden. \\
Soortgelijke conclusies kon men trekken over de afbeeldingen, het waren niet de pixels, die er toe deden, maar eerder de context rond de afbeelding.\\
\\
\textit{Wat de auteur bleek te verbazen, leek voor mij doodnormaal. Het leek mij normaal dat randvoorwaarden dat de meta-data veel meer uitmaakten. Om maar een voorbeeld aan te halen, de rating systeem van Netflix: wanneer ik zelf op zoek ga naar een nieuwe film om te bekijken, dan laat ik mij ook niet  direct be\"{i}nvloeden door de content van de film, maar bijvoorbeeld door hoe recent de film is, hoe de gebruikers-reviews zijn of hoe vaak het bekeken werd, dit zijn allemaal meta-data die de content beschrijven en die meer vertellen over de content dan de content zelf. Het illustreert wel heel mooi het nut van geziene technieken zoals het semantisch web. Ik onderschrijf ook de conclusies van de auteur volledig, men zou steeeds naar de content moet kijken en deze analyseren, maar elke zichzelf respecterende wetenschapper zou ook de context(=meta-data!) rond een content moeten in rekening brengen.}
\end{document}


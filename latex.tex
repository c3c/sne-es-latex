\section{\LaTeX{}}
\label{sect:latex}

\subsection{Tex vs. Word vs. Writer}

\begin{cquote}{Cedric}
The article goes into detail about how LaTeX compares to MS Word en OO Writer. In my opinion, it is a bit short-sighted to only state Latex's typesetting features. Of course it has a bigger flexibility compared to the alternatives, but this doesn't mean that it outperforms its competitors qua usability. LaTeX offers extensive control over the display of the document and fonts. For instance, control over ligatures, and even control over the spacing between ellipsis dots. It is nice that you CAN do these things with LaTeX, but the situations in which I actually needed these features are limited to a handful. 99\% of the day to day use doesn't need this extensive control and in these cases I would favor using WYSIWYG editors. If I have to make a report with a lot of mathematical equations I could imagine LaTeX being my tool of choice.

\vspace{10pt}
Word offers themeing and templating features that are available at the distance of a mouse click. If used correctly, it is a powerful tool. New users are inclined to apply their formatting manually though, which brings inconsistencies along.
\end{cquote}


\begin{cquote}{Leendert}
Latex wins most friendly features, with Word and Writer in a tie. Latex has an edge with larger documents, being plain text input. Word and Writer have the power of integrating Spreadsheets and other 'documents' which can be quite useful.

\vspace{10pt}
A Latex document is like a program, and a Word/Writer document is easy to get 'good' output from, but very hard to get 'better than good' output. With all the control you have in Latex it might be a tad harder to get the output you want, but once there it tends to be magnificent.

A comparison is made in table \ref{tab:LatexComparison}.
\end{cquote}

\subsection{Word vs. LaTeX}

\begin{cquote}{Cedric}
The second article compares MS Word and LaTeX also, but has a few errors in it: Word does have bibliography and citation features on board, although stated otherwise. Compatibility problems become less and less of an issue for MS Word, as it also supports the open document format nowadays, and its docx extension is basically a zipped version of an XML file with formatting applied to it. It is true that only experienced users know how to make good use of Word's layout features, but compared to the learning curve of LaTeX, I think the attributed stars to both cancel each other out. I have to agree though on the price and availability part.
\end{cquote}

\begin{cquote}{Leendert}
Small documents ease, goes to Word. Setting the simple markup is just easy and tweakable. Once you have larger documents the Word markup can be good, but in general Latex markup is advanced and can be styled as you want it.

\vspace{10pt}
When handling larger documents Word can become unwieldy, slow. Latex only processes it when you are done, and can get something to eat. Not forcing you to wait during your moment of inspiration.

\vspace{10pt}
If you want a simple document word has the best learning curve, type and a few clicks get your result. Latex however requires basic knowledge of Latex prior to getting results. Once you have the knowledge most features can be used more efficiently than in Word's click model.

\vspace{10pt}
Cost of ownership is in favor of Latex, being free and open. This also goes for the easily integrated citation tools.

\vspace{10pt}
Storing the results in Latex is often PDF, which is widespread, and not dependent on one vendor like the default Word output.

\vspace{10pt}
In general, if you know neither, and want a short document grab Word. If you want a Long document, even if you don't know any Latex, investing the time will probably be worth it. And once you know the basics, just pick what suits you.
\end{cquote}

\subsection{Why should you use Latex}

\begin{cquote}{Cedric}
The guy on stackexchange sums it up quite well: you shouldn't use LaTeX if you don't have time to learn it, or if the design is more important than the typesetting of the document. I do like how the content of the document is separated from the formatting of the document. Although newer versions of Word also use XML documents for the content, while storing the formatting elsewhere.
\end{cquote}


\begin{cquote}{Leendert}
It is portable, human readable file format (any text editor can be used to salvage a document, even if you don't have any Latex/related tools). Try doing that with word.

\vspace{10pt}
You separate styling and content, keeping the content cleaner and styling reusable.
\end{cquote}

\subsection{Writing a thesis: Word or Latex}

\begin{cquote}{Cedric}
The narrator felt frustrated about Word, because Word didn't do what he expected it to do. He says that LaTeX is good at what Word is bad at (and I agree). Although bibliographical features are available in Word, you have much more control over them in LaTeX.
\end{cquote}

\begin{cquote}{Leendert}
Word is complex, hard, impish Latex has advanced automatic features, easily invoked.

\vspace{10pt}
Style is Latex's thing, it does it good. Word does a lot for you, which it can decide to do whenever you don't want it. Saving, messing up intended layout and more.

\vspace{10pt}
The method of writing differs, Latex takes getting used to. Borrow an example, learn Latex the easy way. It's worth it.
\end{cquote}

\begin{table}[h]
\begin{center}
	\begin{tabular}{ | l | l | l | l | }
		\hline
		Feature/support for & Word & Writer & LaTeX \\
		\hline
		Small caps	&2	& 1	& 3 \\
		Proper numbers in text&	 1	& 1	& 3\\
		Ligatures	& 2	 &1	& 3\\
		Badness	& 1	& 1	& 3\\
		Hyphenation	& US	& US	&GB\\
		Drop cap	 &2	 &2	& 3\\
		\ldots	& 1	& 1	& 3\\
		\hline
	\end{tabular}
	\end{center}
	\caption{A comparison of LaTeX against the \emph{others}}
	\label{tab:LatexComparison}
\end{table}	
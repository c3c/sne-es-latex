
\section{Git and SVN}

\subsection{The polite programmer presents - Torvalds Vs the world of CVS}

\[
Git >  Mercurial > tar+patches > SVN > CVS
\label{math:linus-svn}
\]

Linus T. preferred source management methods are shown in relation \ref{math:linus-svn}.


\subsubsection{History, Bitkeeper}
An early scm, decent but why settle for less than perfect.
Git's flow is partially based on Bitkeeper but this was commercially distributed system which people did not like and could not even try to improve.

\subsubsection{Future, what features would a replacement have?}

\begin{enumerate}
  \item What goes in, comes out. Undetected corruption is not an option.
  \item Performance, if it takes too much time\ldots
  \item Merging easy, Branching is trivial. Merging should be easy.
  \item Diff, to show what and how much changed. But why only on files?
\end{enumerate}

\subsubsection{A distributed system}
What are possible motives for switching to a distributed system,
\begin{itemize}
 \item Centralized collaboration is hard due to branch cost, more accuratly merging branches.
 \item You dont\'t want a central location waiting to fail or be compromized.
 \item You don\'t want any single point of failure.
 \item You can work offline for longer periods of time.
 \item By distributing you have a crude form of backup as all copies can be used equally.
\end{itemize}

However, there are some properties a distributed system has which reduce the fun of building, using and maintaining it,
for example branching is required for any (offline) use. While the branching can
be rather cheap in terms of storage and computational requirements, the demand
for constant merging will require special care and introduce overhead in using
the system.


Using a distributed system you can benefit from a Trust network, by modelling
the repositories to be compatible you can easilly incorperate/pull code and patches
from those peers you trust, and by extension and smart merging, the peers they
trust. Each person can determine who has access to their own repository by means
of their own choice. A use case for this would be a realease team that can read
from a developers repository\footnote{but not write if they choose to} and
carefully test all changes they introduce into a 'verified' release grade
repository.

A team of developers, one of potentially many, can share their branches and
code without continously running into conflicts with other teams. Therefore enabling them to
develop complex experimental features collaboratively, while maintaining a stable codebase for
other teams who might work on their possibly related or even conflicting features.


A decent distributed code management system will allow a user to track the source and history of code, features and bugs.
By making sure a full history is available you can enjoy the power of quality reviews, or audits should someone expect foul play.


\subsubsection{The powers of Git}

\begin{itemize}
\item
Even if you don't care about Git and just adore CVS\footnote{Do not let L. Torvalds know this, for your own sake}
you can use Git just for merging, then put it all back in CVS.

\item
Git makes merging easier than the competition by automating what it can.

\item
Merging can be delegated to the best knowing party, in the case of a conflict the person who caused it will know best what to do.

\item
You dont need to see or share all existing branches, a developer work in a private experimental branch and push it out when done or decent.

\item
Branch names need not be unique across the globe so you can use natural names

\item
A 'super' project can contain references to multiple projects which can be shared, much like libraries of code.

\item
Git checksums everything, cryptographically secure, this to ensure the data you put in you also get out. It makes it harder to sabotage the history with nasty suprises.

\item
Trust your data to be valid, Git does all it can to detect corruption for you.

\item Trust the data even from a shaky source, as long as you have the checksum from another more reliable source.


\item
Git can track full history of changes, however if you really need to you can rewrite it, but you can't really hide the act.

\item 
Tracks history regardless of the files the content lives in, tracking history even if you move a piece of text arround.

\end{itemize}


\subsection{Hg Vs Git}
\label{sect:rebase}

%Figure of a mercurry blob vs an idiot


When comparing Mercurial\footnote{Also known as Hg} with Git you can run into these statement, each relevant to selecting the system suited for you.
\begin{itemize}
\item
Hg history is sacred, you cannot change it.

\item
Git lets you rebase/change/destroy history 'easilly' if you explicitly instruct it to.

\item
Git does not really cater to SVN users, by design.

\item
There are plugins for both, with Hg having a tad more windows fiendly software, and Linux being the most efficient platform for Git.
If you use Eclipse just ignore the operating system, it has plugins for everything and anything.

\item
Hg cares about windows, which could be usefull when you run into a bug or want the last bit of performance you want.
Git can be run on windows at a small performance reduction when compared to Linux, but Git is more pleasant in a Linux environment.

\item
Hg might be more backwards compatible\footnote{Wikipdia style, Citation needed}, though we do have this on authority.

\item
Hg can support shell written extensions using alliasses, allowing core features to be accessed. This is a powerfull way to write plugins to extend the power of Hg.

\item
There are always options to buy commercial support for Hg.
Though Git will also have it's share of knowledgable consultants\footnote{Citition desired}.

\item
Hosting a git or Hg server can be done or done for you easilly at affordable fees.

\end{itemize}


Why could you choose Git as the best system for You?
\begin{enumerate}
\item
Things in git are immutable, being able to revert changes and undo mistakes is a useful safety net.

\item
Git history is safe and stored indefinately, unless you go to the trouble of telling Git to like in Section \ref{sect:rebase}
Rewriting history can be done in both Hg and Git, the Git way of doing so being quite powerful and once you know how, easier.

\item
Bragging rights, just because \textit{Branch} sounds cooler than \textit{bookmark}, and is actually a \textcolor{red}{core} feature.

\item
Git allows partial files to be committed, leaving some changes local if wanted (Such as a 'comment to self, spellcheck my name').
This still holds even for larger files, where it becomes much more useful.

\item
\large{\textit{git blame -C -s}}
Track ownership/history even if you move a feature or bug from \textcolor{green}{a.file} to \textcolor{blue}{b.file}.


\end{enumerate}


\subsubsection{A simple comparisson of VCS's}
There is always one rule to manage by;
Do not anger the programmer, choose the right DVCS

\begin{tabular}{|l|c|c|c|c|}
\hline
Feature             & CVS & SVN & Git & Hg\\
\hline
Mature              & +++ & ++  &  +  & + \\
Track moved content &  -  &  -  & +++ & + \\ %?hg svn
Atomical operations &  -  &  +  &  +  & + \\ 
Branching possible  &  +  &  +  &  +  & + \\
Branching pleasant  &  -  &  -  & +++ & + \\
P2P                 &  -  &  -  & +++ & ++\\ %? hg
Speed               &  -  &  -  & +++ & - \\
Offline history     &  -  &  -  &  +  & + \\
Windows compatible  &  +  &  +  &  +  &+++\\
Learning curve      &  +  &  +  &  -  & + \\
Documentation       &  +  &  +  &  +  & + \\ %? they are all documented, howto'ized etc.
2 parent merge      &  -  &  -  &  +  & - \\
Extensions          &  -  &  -  &  -  & + \\
Core power          &  -  &  -  &  +  & - \\
Small hosting cost  & low & low & low &low\\
Huge hosting cost   &High &High &low&low\\
\end{tabular}


Of course, learning to use any new system is bound to slow down progress for a while, so why not consider 
keeping the system that has been in place for a while now? Are you running into any problems? Changing VCS
will manage cause at least some problems.

